%-------------------------------------------------------------------------------
% GLOSSARY ENTRIES
%-------------------------------------------------------------------------------
\newglossaryentry{PFA}{
name=PFA,
description={Notre project bien modeste}
}
\newglossaryentry{latex}{
name=LaTeX,
description={Un système de préparation de documents de haute qualité}
}
\newglossaryentry{glossaire}{
name=Glossaire,
description={Une liste alphabétique de termes et de leurs définitions}
}
\newglossaryentry{poo}{
name=Programmation Orientée Objet (POO),
description={Un paradigme de programmation qui utilise des objets pour organiser et structurer le code.}
}
\newglossaryentry{api}{
name=Interface de Programmation d'Application (API),
description={Un ensemble de définitions et de protocoles permettant à des applications de communiquer entre elles.}
}
\newglossaryentry{json}{
name=JavaScript Object Notation (JSON),
description={Un format de données textuelles utilisé pour représenter des objets structurés.}
}
\newglossaryentry{yaml}{
name=YAML Ain't Markup Language (YAML),
description={Un format de sérialisation de données lisible par l'homme.}
}
\newglossaryentry{xml}{
name=Extensible Markup Language (XML),
description={Un langage de balisage qui définit des règles pour l'encodage de documents dans un format lisible par l'homme et les machines.}
}
\newglossaryentry{swot}{
    name= Analyse SWOT,
    description={Une technique de planification stratégique utilisée pour identifier les forces, les faiblesses, les opportunités et les menaces d'une organisation ou d'un projet.}
}
\newglossaryentry{rest}{
name= REST,
description={Un style architectural pour la conception d'applications web et d'API.}
}
\newglossaryentry{django}{
name=Django,
description={Un framework web Python de haut niveau qui encourage le développement rapide et la conception pragmatique.}
}
\newglossaryentry{react}{
name=React,
description={Une bibliothèque JavaScript déclarative, efficace et flexible pour la création d'interfaces utilisateur.}
}
\newglossaryentry{nextjs}{
name=Next.js,
description={Un framework React pour la création d'applications web rapides et conviviales pour les utilisateurs.}
}
\newglossaryentry{postgresql}{
name=PostgreSQL,
description={Un système de gestion de base de données relationnelle objet open source puissant.}
}
\newglossaryentry{redis}{
name=Redis,
description={Un magasin de structures de données en mémoire open source, utilisé comme base de données, cache, broker de messages et file d'attente de flux.}
}
\newglossaryentry{celery}{
name=Celery,
description={Un framework de file d'attente de tâches distribuées open source pour exécuter des tâches asynchrones en arrière-plan.}
}
\newglossaryentry{textblob}{
name=TextBlob,
description={Une bibliothèque Python pour le traitement de données textuelles.}
}
\newglossaryentry{materialui}{
name=Material-UI,
description={Une bibliothèque de composants React qui implémente Google's Material Design.}
}
\newglossaryentry{tailwindcss}{
name=Tailwind CSS,
description={Un framework CSS utilitaire pour créer des designs personnalisés rapidement.}
}
\newglossaryentry{redux}{
name=Redux,
description={Une bibliothèque JavaScript pour gérer l'état des applications, principalement utilisée avec React.}
}
\newglossaryentry{axios}{
name=Axios,
description={Une bibliothèque JavaScript basée sur des instructions pour effectuer des requêtes HTTP.}
}
\newglossaryentry{socketio}{
name=Socket.IO,
description={Une bibliothèque JavaScript pour la communication bidirectionnelle et en temps réel entre le navigateur web et le serveur.}
}
\newglossaryentry{jwt}{
name=JSON Web Token (JWT),
description={Une norme ouverte pour transmettre des informations de manière sécurisée entre les parties en tant qu'objet JSON.}
}
\newglossaryentry{oauth}{
name=OAuth,
description={Un protocole standard ouvert pour l'autorisation, couramment utilisé comme une méthode permettant aux utilisateurs d'Internet de se connecter à des sites Web tiers.}
}
\newglossaryentry{IA}{
name=Intelligence Artificielle (IA),
description={L'intelligence démontrée par les machines, par opposition à l'intelligence naturelle affichée par les êtres vivants.}
}
\newglossaryentry{machinelearning}{
name=Apprentissage Automatique (Machine Learning),
description={Une branche de l'intelligence artificielle (IA) et de l'informatique qui se concentre sur l'utilisation de données et d'algorithmes pour imiter la façon dont les humains apprennent, en améliorant progressivement sa précision.}
}
\newglossaryentry{apprentissageprofond}{
name=Apprentissage Profond (Deep Learning),
description={Un type d'apprentissage automatique et d'intelligence artificielle (IA) qui imite la façon dont les humains gagnent certains types de connaissances.}
}
\newglossaryentry{gamification}{
name=Gamification,
description={L'application d'éléments de conception de jeu et de principes de jeu dans des contextes non liés au jeu.}
}
\newglossaryentry{accessibilite}{
name=Accessibilité,
description={La pratique consistant à rendre vos sites Web utilisables par le plus grand nombre de personnes possible.}
}
\newglossaryentry{internationalisation}{
name=Internationalisation,
description={Le processus de conception et de développement d'un produit, d'une application ou d'un contenu de document qui permet une localisation pour des groupes cibles qui varient en culture, région ou langue.}
}
\newglossaryentry{localisation}{
name=Localisation,
description={L'adaptation d'un produit ou d'un contenu à un emplacement ou un marché géographique ou culturel spécifique.}
}
\newglossaryentry{api_endpoint}{
name=Point de Terminaison d'API (API Endpoint),
description={Un point de contact pour un système ou un service qui communique avec un autre système ou service.}
}
\newglossaryentry{serialisation}{
name=Sérialisation,
description={Le processus de traduction des structures de données ou des états d'objets en un format qui peut être stocké (par exemple, dans un fichier ou une mémoire tampon de données) ou transmis et reconstruit ultérieurement.}
}
\newglossaryentry{deserialisation}{
name=Désérialisation,
description={Le processus de conversion d'un flux de données (par exemple, à partir d'un fichier ou d'une mémoire tampon) en une structure de données ou un état d'objet.}
}
\newglossaryentry{unit_tests}{
name=Tests Unitaires (Unit Tests),
description={Des tests effectués sur de petites unités de code pour vérifier leur fonctionnement.}
}
\newglossaryentry{typescript}{
name=TypeScript,
description={Un langage de programmation open source développé par Microsoft qui ajoute des types à JavaScript.}
}
\newglossaryentry{llm}{
name=Large Language Model,
description={Un modèle de langage qui est capable de générer du texte de manière autonome.}
}
\newglossaryentry{django_filters}{
name=Django Filters,
description={Une application Django réutilisable pour la gestion des filtres de requête.}
}
\newglossaryentry{django_extensions}{
name=Django Extensions,
description={Une collection d'extensions personnalisées pour le framework Django.}
}
\newglossaryentry{django_redis}{
name=Django Redis,
description={Un backend de cache/session Redis complet pour Django.}
}
\newglossaryentry{corsheaders}{
name=CORS Headers,  
description={Une application Django pour gérer les en-têtes de serveur requis pour le partage de ressources entre origines (CORS).}
}
\newglossaryentry{whitenoise}{
name=WhiteNoise,
description={Permet à votre application Django de servir ses propres fichiers statiques, facilitant le déploiement d'applications Django sans dépendre de nginx, Amazon S3 ou tout autre service externe.}
}
\newglossaryentry{django_channels}{
name=Django Channels,
description={Fournit à Django un support pour les WebSockets de manière similaire aux vues HTTP traditionnelles.}
}
\newglossaryentry{simplejwt}{
name=Simple JWT,
description={Un plugin d'authentification JSON Web Token pour le Django REST Framework.}
}
\newglossaryentry{django_allauth}{
name=Django Allauth,
description={Fournit l'authentification, l'enregistrement, la gestion de compte ainsi que l'authentification de compte tiers (sociale).}
}
\newglossaryentry{drf_spectacular}{
name=DRF Spectacular,
description={Un outil de génération de schéma pour Django REST Framework qui génère une documentation d'API moderne.}
}
\newglossaryentry{formik}{
name=Formik,
description={Une bibliothèque de gestion de formulaires pour React qui facilite la gestion des formulaires.}
}
\newglossaryentry{yup}{
name=Yup,
description={Une bibliothèque de validation de schéma JavaScript pour les valeurs.}
}
\newglossaryentry{reactrouter}{
name=React Router,
description={Une bibliothèque de routage pour React qui permet de naviguer entre les composants React.}
}
\newglossaryentry{reacthookform}{
name=React Hook Form,
description={Une bibliothèque de gestion de formulaires pour React qui utilise des hooks pour gérer les formulaires.}
}
\newglossaryentry{reactuse}{
name=React Use,
description={Une collection de hooks personnalisés pour React.}
}
\newglossaryentry{reactplayer}{
name=React Player,
description={Un lecteur multimédia pour React.}
}
\newglossaryentry{reactspring}{
name=React Spring,
description={Une bibliothèque d'animation pour React qui permet de créer des animations fluides.}
}
\newglossaryentry{reactselect}{
name=React Select,
description={Une bibliothèque de sélection pour React qui permet de créer des champs de sélection personnalisés.}
}
\newglossaryentry{reactquill}{
name=React Quill,
description={Un éditeur de texte riche pour React.}
}
\newglossaryentry{reactmarkdown}{
name=React Markdown,
description={Un composant Markdown pour React.}
}
\newglossaryentry{reactpdf}{
name=React PDF,
description={Un générateur de PDF pour React.}
}
\newglossaryentry{reactchartjs}{
name=React Chart.js,
description={Une bibliothèque de graphiques pour React qui permet de créer des graphiques interactifs.}
}
\newglossaryentry{reactquery}{
name=React Query,
description={Une bibliothèque de gestion de cache pour React qui fournit des fonctionnalités de requête et de mutation de données.}
}
